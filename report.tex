\documentclass[10pt, A4paper]{article}

\usepackage{amsmath}
\usepackage{amssymb}
\usepackage{amsthm}
\usepackage{mathtools}
\usepackage[linesnumbered,ruled]{algorithm2e}

\usepackage{natbib}
\setcitestyle{numbers, square}

\newtheorem{theorem}{Theorem}[section]
\newtheorem{lemma}[theorem]{Lemma}
\newtheorem{corollary}[theorem]{Corollary}

\DeclareMathOperator{\diag}{diag}
\DeclareMathOperator{\rank}{rank}

%%%%%%%%%%%%%%%%%%%%%%%%%%%%%%%%%%%%%%%%%%%%%%%%%%%%%%%%%%%%%%%%%%%%%%

\begin{document}

\title{Polar Decomposition}
\author{Thomas Seleiro}
\maketitle


\begin{enumerate}
	\setcounter{enumi}{1}
	\item Prove that the singular values of $A$ are the eigenvalues of $H$.
\end{enumerate}

We know that any matrix $A\in\mathbb{C}^{m \times n}$, $m\geq n$ has 
a thin singular value decomposition $A = P \Sigma Q^*$ where $P \in
\mathbb{C}^{m \times n}$ has orthogonal columns, $Q \in 
\mathbb{C}^{n \times n}$ is unitary, and $\Sigma \in 
\mathbb{C}^{n \times n}$ is diagonal with $\Sigma = \diag(\sigma_1, 
\ldots, \sigma_r)$ where $\rank(A) = r$ and $\sigma_1 \geq \ldots \geq
\sigma_r \geq 0$, the singular values of $A$. Thus we can write
\begin{align}
	A = (PQ^*) (Q \Sigma Q^*) \eqqcolon UH
	\label{eq:PolarSVD}
\end{align}
where $U$ and $H$ satisfy the properties of a polar decomposition.

In particular we have $H = Q \Sigma Q^*$ where $\Sigma$ is diagonal and 
$Q$ is orthogonal. Thus the diagonal values of $\Sigma$ are the 
eigenvalues of $H$, which are the singular values of $A$.


\vspace{0.2cm}
\begin{enumerate}
	\setcounter{enumi}{2}
	\item Prove that $A$ is normal ($A^*A = AA^*$) iff $U$ and $H$
	commute.
\end{enumerate}

We first suppose that $U$ and $H$ commute. Note that for the product
$HU$ to be well defined, we must have $m = n$ which implies
$U\in\mathbb{C}$ is unitary. Since $A = UH = HU$ we get
\begin{align}
	\label{eq:AstarA}
	A^*A &= (UH)^* (UH) = H^*(U^*U)H = H^2 \\
	AA^* &= (HU) (HU)^* = H(UU^*)H^* = H^2
\end{align}
so $A$ is normal.

Now suppose $A$ is normal. Since $A^*A \in \mathbb{C}^{n\times n}$ and
$AA^* \in \mathbb{C}^{m \times m}$, $A$ normal requires $m=n$.
Using the singular value decomposition of $A$, we have
\begin{align}
	AA^* = (P\Sigma Q^*) (Q\Sigma P^*) = P\Sigma^2 P^*
	\label{eq:AAstar}
\end{align}
where $\Sigma^2 = \diag(\sigma_1^2, \ldots,\sigma_r^2)$.
Equating (\ref{eq:AstarA}) and (\ref{eq:AAstar}), we get 
$H^2 = P\Sigma^2P^*$. From~[\citealp{hojo1985}, p.405], we know 
that there
is a unique Hermitian positive semi-definite matrix $(AA^*)^{1/2}$ such 
that $(AA^*)^{1/2}(AA^*)^{1/2} = AA^* = H^2$.
It is obvious by its construction that $H$ is said matrix, but we also
note that $(P\Sigma P^*) (P\Sigma P^*) = P \Sigma^2P^* = AA^*$.
Therefore $H = P\Sigma P^*$ and
\begin{align}
	HU = (P\Sigma P^*) (PQ^*) = P \Sigma Q^* = A = UH
\end{align}
by the properties of the SVD of $A$. Therefore $U$ and $H$ commute.
 

\vspace{0.2cm}
\begin{enumerate}
	\setcounter{enumi}{3}
	\item Verify the formula
	\begin{align*}
		U = \frac{2}{\pi}A \int_{0}^{\infty} (t^2I - A^*A)^{-1}dt
		\tag{*}
		\label{eq:Q4}
	\end{align*}
	for full rank $A$ by using the singular value decomposition (SVD)
	of $A$ to diagonalize the formula.
\end{enumerate}

Since $A^*A = (Q\Sigma P^*)(P\Sigma Q^*) = Q\Sigma^2Q^*$, we have
\begin{align}
	t^2I + A^*A = Q(t^2I)Q^* + Q \Sigma^2Q^* = QDQ^*
	\label{eq:Q4eq1}
\end{align}
where $D \coloneqq \diag(t^2 + \sigma_i)$.
Inverting (\ref{eq:Q4eq1}) gives
\begin{align}
	(t^2I + A^*A)^{-1} = QD^{-1}Q^*, \qquad
	D^{-1} = \diag\left(\frac{1}{\sigma_i^2 + t^2} \right)
\end{align}
Since $Q$ and $Q^*$ do not depend on $t$, they can be taken outside the 
integral, leaving the right hand side of~(\ref{eq:Q4}) in the form
\begin{align}
	\frac{2}{\pi} A\, Q\int_{0}^{\infty}D^{-1}dt\, Q^*.
\end{align}
The integral is a diagonal matrix where the $i$th diagonal component is
\begin{align}
	\int_{0}^{\infty} \frac{1}{\sigma_i^2 + t^2} \, dt =
	\left[\frac{1}{\sigma_i} \arctan \left(\frac{t}{\sigma_i}\right)
	\right]_0^{\infty} = \frac{\pi}{2\sigma_i}
\end{align}
using~[\citealp{jeda2008}, 4.2.4.4]. So the right-hand side of 
(\ref{eq:Q4}) is
\begin{align}
	\begin{split}
	A\,Q\diag\left(\sigma_i^{-1}\right)Q^*&=
	P\Sigma Q^* \, Q \Sigma^{-1}Q^* = PQ^* = U 
%	\Sigma^{-1} isn't necessarily well defined
	\end{split}
\end{align}


\vspace{0.2cm}
\begin{enumerate}
	\setcounter{enumi}{4}
	\item Derive Newton's method for computing U by considering
	equations $(X+E)*(X+E) = I$, where $E$ is a ``small perturbation''.
	(Newton's method is $X_{k+1} = (X_k + X_k^{-*})/2, X_0 = A$)
\end{enumerate}

We know that $U$ is the closest unitary matrix to $A$, and since 
$U^*U=I$, we try to find a solution to the equation
\begin{align}
	F(X) = 0,\qquad F(X)\coloneqq X^*X - I
\end{align}
using a Newton method starting at $A$.
The general form of the Newton method [\cite{Kell2003}, p.] is
\begin{align}
	F(X_{k+1}) + DF_{X_k} \left[X_{k+1} - X_k\right] = 0
	\label{eq:genNewt}
\end{align}
where $DF_{X_k}$ is the Fréchet derivative and, is the first order $E$ 
term in
\begin{align}
	F(X+E) - F(X) = X^*E + E^*X + E^*E.
\end{align}
So $DF_{X_k}[E] = X^*E + E^*X$. Substituting in~(\ref{eq:genNewt}),
\begin{align}
	X_k^*X_k - I + X_k^* \left( X_{k+1} - X_k \right) + 
		\left( X_{k+1}^* - X_k^* \right)X_k &= 0 \\
	X_k^*X_k - I + X_k^* X_{k+1} - X_k^* X_k + 
		X_{k+1}^* X_k - X_k^*  X_k &= 0 \\
	X_k^* X_{k+1} + X_{k+1}^*X_k &= X_k^* X_k + I
\end{align}
We know that for any matrix we can write $B = 1/2(B + B^*) + 
1/2(B-B^*)$, where the terms on the right-hand side are the Hermitian 
and skew Hermitian components respectively [\citealp{hojo1985}, p.170].
Setting the skew Hermitian part to zero, and taking $B = X_k^*X_{k+1}$ 
gives
\begin{align}
	X_k^*X_{k+1} &= \frac{1}{2} \, (X_k^*X_k + I) \\
	X_{k+1} &= \frac{1}{2} \, (X_k + X_k^{-*})
	\label{eq:newton}
\end{align}


\vspace{0.2cm}
\begin{enumerate}
	\setcounter{enumi}{5}
	\item Prove that Newton's method converges, and at a quadratic
	rate, by using the SVD of $A$.
\end{enumerate}

For the Newton iteration to be well defined, we require that $A$ and 
the iterates $X_k$ be invertible.

We have the SVD of $A = P\Sigma Q^*$ and $U = PQ^*$.
The iterates $X_k$ also have a singular value decomposition, which we 
write $X_k=P_k \Sigma_k Q_k^*$. Using this in eq.~(\ref{eq:newton}) 
gives
\begin{align}
	X_{k+1} = (X_k + X_k^{-*})/2 &= \frac{1}{2} (P_k \Sigma_k Q_k^* + 
	P_k \Sigma_k^{-1} Q_k^*)\\
	&= P_k \frac{1}{2}(\Sigma_k + \Sigma_k^{-1}) Q_k^*
\end{align}
So we can identify the factors in the SVD of $X_{k+1}$ (up to 
reordering of rows) and get 
\begin{align}
	P_k = P, \qquad Q_k = Q, \qquad \Sigma_{k+1} = \frac{1}{2}
	\left(\Sigma_k + \Sigma_k^{-1} \right)
\end{align}

We now have
\begin{align}
	U - X_{k+1} &= PQ^* - P\left[\frac{1}{2}
	\left(\Sigma_k + \Sigma_k^{-1}\right)\right] Q^* \\
	&= \frac{1}{2} P \left[\left(I - \Sigma_k\right) + 
	\left(I - \Sigma_k^{-1}\right) \right] Q^* \\
\end{align}
And since
\begin{align}
	- \Sigma_k^{-1} (I - \Sigma_k)^2 &= -\Sigma_k^{-1}
	\left(I - 2 \Sigma_k + \Sigma_k^2\right) \\
	&= 2I -\Sigma_k - \Sigma_k^{-1}
\end{align}
we are left with $U - X_{k+1} = -P \Sigma_k^{-1} \left( I - \Sigma_k
\right)^2 Q^* /2$.
Taking the 2-norm on both sides and exploiting the fact that $P$ and 
$Q$ are orthogonal,
\begin{align}
	\| U - X_{k+1} \|_2 &\leq \frac{1}{2}\, \left\| P\Sigma_k^{-1} 
	\right\|_2 \left\| (I - \Sigma_k)^2 Q^*\right\|_2 \\
	&= \frac{1}{2}\, \left\|\Sigma_k^{-1}\right\|_2 \left\| \left( I - 
	\Sigma_k \right)^2 \right\|_2 \\
	&\leq \frac{1}{2}\, \|X_k\|_2 \|I - \Sigma_k\|_2^2 \\
	&= \frac{1}{2}\, \|X_k\|_2 \|U - X_k\|_2^2
\end{align}

To achieve quadratic convergence, we need to bound $\|X_k\|$ by a 
constant. We do so by observing that
\begin{align}
	\|X_{k+1}\|_2 = \max_{i = 1:n} \frac{\sigma_i + \sigma_i^{-1}}{2}
	\leq \max \left\{ \left\|X_k \right\|_2
	, \left\|X_k^{-1} \right\|_2 \right\}
\end{align}
and so for all $k$, $\|X_k\|_2 \leq M \coloneqq \max\{\|A\|_2, 
\|A^{-1}\|_2\}$.
Thus we can conclude that the Newton method converges quadratically.



\vspace{0.2cm}
\begin{enumerate}
	\setcounter{enumi}{6}
	\item Use the SVD to analyse the convergence of the
	Newton\nobreakdash-Schulz iteration for computing $U$:
	\begin{align*}
		X_{k+1} = \frac{1}{2}X_k(3I - X_k^* X_k), \qquad X_0 = A
	\end{align*}
\end{enumerate}


\vspace{0.2cm}
\begin{enumerate}
	\setcounter{enumi}{7}
	\item Evaluate the operation count for one step of Newton's 
	method and one step of the Newton\nobreakdash-Schulz iteration (taking
	account of symmetry). Ignoring operation counts, how much faster
	does matrix multiplication have to be than matrix inversion for
	Newton\nobreakdash-Schulz to be faster than Newton (assuming both take the same number of iterations)?
\end{enumerate}

We first consider the $k$th step $X_{k+1} = (X_k + X_k^{-*})/2\,$ for 
the Newton iteration of a matrix $A \in \mathbb{C}^{n\times n}$. We 
identify 3 main operations:
\begin{itemize}
	\item One matrix inversion of $X_k \in \mathbb{C}^{n \times n}$: 
	\hfill
	$2n^3 + O(n^2)$ flops
	
	\item One matrix addition in $\mathbb{C}^{n \times n}$:
	\hfill
	$n^2$ flops
	
	\item One element-wise division in $\mathbb{C}^{n \times n}$:
	\hfill
	$n^2$ flops
\end{itemize}
So the total number of operations for the Newton method is $2n^3 + 
O(n^2)$.

We now consider a step of the Newton-Schulz iteration $X_k(3I - 
X_k^*X_K)/2$ for $X_0 = A \in \mathbb{C}^{m \times n}$.
We fist note that $X_k^*X_k$, and by extension $(3I - X_K^*X_k)/2$, is 
Hermitian in $\mathbb{C}^{n \times n}$.
Therefore only the upper triangular elements of these matrices need to 
be calculated, ie we only have to compute $\sum_{k=1}^{n} k= (n^2+n)/2$ 
components.
Calculating $X_k*X_k$, we use the Algorithm~\ref{alg:XstarX},
\begin{algorithm}
	$b_{ij} = (X_k^*X_k)_{ij}$ \;
	\For{$i = 1:n$}
	{
		\For{$j = i:n$}
		{
			\For{$r = 1:m$}
				{
					$b_{ij} = b_{ij} + \overline{x_{ri}}x_{rj}$ \;
				}
		}
	}
	\caption{Algorithm to compute the top diagonal elements of 
	$X_k^*X_k$}
	\label{alg:XstarX}
\end{algorithm}
using a total of $mn^2 + mn$ flops. Forming $3I - X_k^*X_k$ and 
dividing the result by 2 adds $n^2 + n$ operations. Finally multiplying 
by $X_k$ takes $2mn^2$ flops for a total of $3mn^2 + O(n^2) + O(mn)$ 
flops per step.



\bibliography{mybib}
\bibliographystyle{plain}


\end{document}